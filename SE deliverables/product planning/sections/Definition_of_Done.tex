\subsection{Definition of Done}
This chapter describes the definition of done so that we both will have the same end goals. This will enable us to know when a feature is done and when the corresponding sprint item can be closed.

Our definition of done focuses on three levels: backlog items (features), sprints and releases.

\paragraph{4.1 Level 1: Backlog items} 
We consider a backlog item as done when it has an test coverage of at least 75\% for non-GUI elements. These tests can be divided into unit tests and other automated tests. For a feature to be merged (using a pull request) into the release version of the product, it has to be approved by the other team member. Their approval will be based on the test coverage, code readability, documentation and the overall code quality. Documentation should be added on every class and method. 

\textbf{4.2 Level 2: Sprints} \\
Every sprint corresponds to one week. At the end of a sprint, all the items on the sprint plan of the current sprint have to be finished, according to the definition of done for backlog items. A new release of our product should be submitted to the version control server (Github) master. All the unit tests should pass and the system should be improved based on added or extended features. There shouldn't be any bugs/errors left in the system and the program should behave and look like the client wanted. We should also have written a sprint reflection for that sprint and a new sprint plan for the coming sprint. 

\textbf{4.3 Level 3: Releases}\\
A release version of our product is done at the end of a sprint. The release version should contain the features that should have been implemented for the corresponding sprint. This also includes the minimal test coverage and other conditions described in the definition of done for a sprint. It is also important to note that the potential feedback of the client for that week should be processed before a release is completed.
In the final release we should have implemented all must haves since the program can't function without these mandatory features. Most should haves (50 to 60\%) and some could haves (30 to 40\%), which are defined in 3.1, have to be implemented. As explained in chapter 3, these features are not necessary for the user and the system to work but our goal is to implement them partially since it would be nice to include them. It would make our program more user friendly. 
Lastly, the final release and other major releases should be approved by the client, the coach and obviously by ourselves too. This means that it should be simple and efficient to use for the client and well documented and designed by us. The SIG test is also an important part of this. They will determine if the code meets the standards and if it is clearly structured and documented. 

As developers we also strive for maintainability and extendability, so that the system can be easily maintained, improved and updated by us or other people after our final release. This will also be taken into account throughout the entire project.

% completion of project: binaries, packaging of the project 
