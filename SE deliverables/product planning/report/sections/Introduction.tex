\section{Introduction}

Careful planning is required to achieve a good development process and a successful software product. This documents provides the prioritized agile backlog of our product and describes the top items. This helps us to determine which items we can and should complete during a particular sprint. By moving items from the product backlog to the sprint backlog, each product backlog is divided into one or more sprint tasks so that we can effectively divide the work and complete our goals. The goal of our project is to create a standalone software application which can be used by researchers from the Air Transport \& Operations department at TU Delft for model-based optimization and visualization of aircraft noise.

In the second chapter of this document an analysis of our program will be provided by describing the context, problem and stakeholder inputs. Based on this the features needed for the product will be described in a high level backlog and the steps needed to complete the project will be defined in a roadmap. Here you will find the major release schedule and goals.

Chapter 3 contains a more detailed product backlog, which is done by means of user stories of features, technical improvements and knowledge acquisition. In chapter 4 a definition of 'done' is given on backlog items, sprints and releases. This will enable us to know when a feature is done and when the corresponding sprint item can be closed. The 5th and last chapter gives a glossary to give you an explanation for any jargon or complicated terms.