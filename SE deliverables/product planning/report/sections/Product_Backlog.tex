\section{Product Backlog}

This chapter describes the product backlog of our program. The first paragraph gives the user stories of the features that need to be implemented. The second paragraph focuses on technical issues found in existing products of competitors which will be solved in our product. The third paragraph describes the user stories for the knowledge acquisition. The fourth and last paragraph shows an initial release plan with milestones. 

\subsection{User stories of features}
In this paragraph the user stories of features are sorted by priority using the MoSCoW method:

\textbf{Must have} \\ 
These features are must haves since the program would not be functioning and meet the (minimal) requirements of the customer without these implemented. 

\begin{itemize}
\item As an ATO researcher, I want to read in an arbitrary data file (.dat extension) or text file indicating the input flight trajectory and grid (based on the RD-coordinate system) that I select from a directory.
\item As an ATO researcher, I want the sound model to calculate four types of noise levels: A-Weighted Sound Exposure Level (SEL), Effective Perceived Noise Level (EPNL), A-Weighted Maximum Sound Level (LAMAX) and Tone-Corrected Maximum Perceived Noise Level (PNLTM). This should be done conform world standards following the AC Model.
\item As an ATO researcher, I want to calculate noise contours (contour area) produced along the input trajectory using noise input data based on six possible noise metrics: single event metrics (SEL, LAMAX, EPNL, PNLTM) and cumulative noise metrics (Day-Evening-Night Average Level $L_{DEN}$, Night Average Level $L_{night}$). This should be done automatically after the noise model has finished calculating the noise levels.
\item As an ATO researcher, I want to export the actual contour data as a text file or .dat file so that I can save the results in a directory. 
\item As an ATO researcher, I want to optimize the input flight trajectory within the International Standard Atmosphere using a dynamic optimization algorithm based on three models: operational constraints (depending on the used aircraft model), noise model (depending on calculated noise contours) and geographical model. If I select the option to optimize, this calculation should be done automatically by the application after the contouring algorithm has been performed.
\item As an ATO researcher, I want to visualize the input flight trajectory and the produced noise contours in a real-time 3D animation mapped on Google Earth.
\item As an ATO researcher, I want to export the resulting animation as a KML file so that I can also run the animation later in Google Earth without needing to start all over again.
\item As an ATO researcher, I want a modern and easy-to-use interface to select files, select the contour(s) I am interested in, select the output that needs to be produced (optimization and/ or visualization) and select a directory to store the output.
\end{itemize}

\newpage

\textbf{Should have}

These features are should haves since the basic components of the program would work without these features but it would be better (meet the requirements of the customer better) if these would be implemented. 

\begin{itemize}
\item As an ATO researcher, I want the sound model to perform real-time.
\item As an ATO researcher, I want to be able to disrupt or cancel the noise, optimization or visualization operations in case I made a mistake and want to change my input data.
\item As an ATO researcher, I want to be able to select a real flight route from the tool FlightRadar24 and visualize this with its calculated noise contours in a real-time 3D animation in Google Earth.
\end{itemize}

\textbf{Could have} \\
These features are could haves since they are not necessary to the customer and to the functioning of the program but it would be nice to include them if there is enough time left in our timeframe. 

\begin{itemize}
\item As an ATO researcher, I want to be able to perform adjustments to the input data in case a data element needs to be changed or added. 
\item As an ATO researcher, I want to have the option to transform the coordinates of my input grid if they are not based on the RD-coordinate system that is required by the noise model.
\item As an ATO researcher, I want to read in Matlab and Excel input files.
\item As an ATO researcher, I want the GUI to show me (real time) how much progress is made in the analysis of the data so that I can estimate how long I have to wait for the results.
\item As an ATO researcher, I want to call the program by a name and recognise it by its logo so that I can share it with others.
\end{itemize}

\textbf{Won't have} \\
These features are won't haves since the customer doesn't actually need these features to be implemented. There probably won't be enough time left for the following features but it could be interesting for a follow-up phase of the project. 

\begin{itemize}
\item As an ATO researcher, I want a logger to log all the operations I have done in the program so that I can have an overview of the adjustments I made.
\item As an ATO researcher, I want to add comments to certain parts of the input data such as data elements and data chunks so that I can make memos of my own reflections on the data.
\end{itemize}

\newpage
\subsection{User stories of technical improvements}
\textbf{Must have} \\
This feature is a must have since it's a main shortage in existing software that the client uses. This technical improvement would be a unique selling point for our program. 
\begin{itemize}
\item As an ATO researcher, I want the program to perform the noise model, optimization model and visualization model in a pipelined fashion and automated manner without me needing to perform any actions (after the input data is entered).
\item As an ATO researcher, I want the program to optimize input trajectories using the contour data and not just the noise data (this algorithm hasn't been implemented before).
\end{itemize}

\textbf{Should have} \\
These features are should haves since their improvement would add (new) functionality to the program but they are not necessary for the customer. 
\begin{itemize}
\item As an ATO researcher, I want the program to be easier to use than the existing NOISHHH tool (noise model) and NoiseLAss program (optimzation model) so that I don't need to spend much time on configurations and learning how to use the program.
\end{itemize}

\subsection{User stories of know-how acquisition}
This paragraph describes which documents we need to read in order to build the background knowledge that is required to make a future technical decision or precise scheduling based on the user's needs. 

We need to read about the following subjects:
\begin{itemize}
\item As an ATO researcher, I want the development team to read the technical manual INM 7.0 for the integrated noise model which represents the algorithms and methodology used to compute noise-level and time-based metrics based upon finite flight-segment data. They can also use this noise tool to validate their results in case they decide to improve the noise model for speed-up.
\item As an ATO researcher, I want the development team to read about the contouring algorithm described in the documentation about the noise assessment tool NoiseLAss. This document explains the general algorithm and the way b-spline interpolation can be used to make contour areas more smooth.
\item As an ATO researcher, I want the development team to read about the Awakenings algorithm (also described in the NoiseLAss documentation) which provides formulas on the calculation of population annoyance caused by aircraft noise.
\item As an ATO researcher, I want the development team to read about the existing trajectory optimization tool NOISHH and the dynamic optimization algorithm it uses (described in Joeri Dons' MSc thesis: 'Optimization of Departure and Arrival Routing for Amsterdam Airport Schiphol'). This way they will understand how to optimize trajectories regarding to the produced noise contours. They can also use this optimization tool to validate their results. 
\end{itemize}

\newpage 

\subsection{Initial release plan (milestones, MRFs per release)}
The milestones are spread across different sprints and described below with the corresponding goals. All the releases are continually tested with system testing throughout the entire project. Please note that a sprint corresponds with one week.

\textbf{Sprint 1: 18/04/2016 - 24/04/2016 } \\
This release will contain at least the following features:
\begin{itemize}
\item A Product Planning document containing a roadmap, overall planning and user stories that are prioritized in consultation with the client.
\item A set-up of the Emergent Architecture document containing a pipeline diagram representing the workflow of our program and the way in which the noise, optimization and visualization models are connected.
\end{itemize}

\textbf{Sprint 2: 25/04/2016 - 01/05/2016 } \\
This release will contain at least the following features:
\begin{itemize}
\item A Research Report in which the problem, context and possible solutions are analysed. This will be discussed with the client.
\item An implementation of the contouring algorithm (refining the grid, finding switch points and clustering points with a similar noise level)
\item An implementation of the algorithm that converts Rijksdriehoekscoördinaten to WGL coordinates (long/lat)
\item Basic visualization of noise contours with an overlay on Google Earth
\end{itemize}

\textbf{Sprint 3: 02/05/2016 - 08/05/2016} \\
This release will contain at least the following features:
\begin{itemize}
\item Extended visualization of the noise contours plotted/ mapped in Google Earth
\item An implementation of the spline interpolation algorithm to smoothen out the contour lines
\item The option to output actual noise data
\item The option to turn on or off particular noise contours in the visualization
\end{itemize}

\textbf{Sprint 4: 09/05/2016 - 15/05/2016} \\
This release will contain at least the following features:
\begin{itemize}
\item Basic implementation of the trajectory optimization model (point-mass calculation)
\end{itemize}

\textbf{Sprint 5: 16/05/2016 - 22/05/2016} \\
This release will contain at least the following features:
\begin{itemize}
\item Full implementation of the trajectory optimization model (added: operational constraints)
\item Visualization of the input trajectory and noise contours in a 3D real-time animation in Google Earth (link all visualization components together)
\end{itemize}

\textbf{Sprint 6: 23/05/2016 - 29/05/2016} \\
This release will contain at least the following features:
\begin{itemize}
\item Speed-up in the trajectory optimization model (real-time)
\item Potential speed-up of the contouring algorithm
\end{itemize}

\textbf{Sprint 7: 30/05/2016 - 05/06/2016} \\
This release will contain at least the following features:
\begin{itemize}
\item A shell script in which all operations are pipelined and performed in an automated manner
\item First version of the GUI containing containing a menu bar and tabs for the import of files, noise model (with raw noise level data as output) and optimization model.
\item An implementation of the Awakenings algorithm to calculate population annoyance
\item Emergent Architecture document presenting the final state of the architecture design. 
\end{itemize}

\textbf{Sprint 8: 06/06/2016 - 12/06/2016} \\
This release will contain at least the following features:
\begin{itemize}
\item The option to insert a real flight route from FlightRadar24 and visualize this with a 3D animation in Google Earth
\item Final version of the GUI representing the three models (noise, optimization, visualization) and all possible user operations (added: visualization tab).
\item A visualization of population annoyance in Google Earth (linked with the animation)
\end{itemize}

\textbf{Sprint 9: 13/06/2016 - 19/06/2016} \\
This release will contain at least the following features:
\begin{itemize}
\item Solved bugs and other problems from the previous sprint(s).
\item Final product containing at least all must-have and should-have requirements.
\item Final report about the developed, implemented, and validated software product.
\end{itemize}