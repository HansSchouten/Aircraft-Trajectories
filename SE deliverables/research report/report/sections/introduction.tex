\section{Introduction}

\paragraph{Target Customers}
Aircraft and airport noise are complex subject matters which have been studied for decades and are still the focus of many research efforts nowadays. Also at the department of Air Transport \& Operations (ATO) at TU Delft’s Faculty of Aerospace Engineering. ATO has three research aims: 1) To develop radical new ways to optimize aircraft operations for efficiency, safety, cost and environmental impact; 2) To extend the analysis to an airline fleet and network level to include capacity and resilience; 3) To synthesize these to include operational safety at an airline and ATM level. To support their research findings at conferences, the researchers at ATO need a standalone application for model-based optimization and visualization of aircraft noise. This report summarizes the research phase of this project, during which we familiarize ourselves with the field and investigate what programming language, libraries, tools, etc. are best suited for our product.

\paragraph{Customer Needs}
We will provide a product that represents a creative and efficient way to minimize and visualize aircraft noise along simulated and real flight routes. This requires the implementation of two mathematical models: one for the computation of noise contours and one for the iterative optimization of aircraft trajectories for minimum noise and population annoyance. The models will be deployed by the research team to predict aircraft noise along a particular trajectory (flight route) and to update the trajectory design in an iterated manner to minimize the produced noise over populated areas around airports. Therefore, the parameters that take part in the optimization of aircraft trajectories range from the generic criterion of contour areas to a number of site-specific criteria based on the impact on population

Additionally, the program should be able to visualize the noise produced along the trajectory through a 3D animation pictured on a real map. This requires a visualization of noise contours, which are ‘noise footprints’ whose shape indicate areas of constant noise. Noise contours are a new subject to the research group and have not been implemented in relation with noise minimization before so this will be a challenging topic for us. The visualization should also show the effects of the produced aircraft noise on population annoyance.

In this orientation report we summarize the results of the research phase of this project as follows: first in Section 2 we give some more detailed information on the applications that we will extend and on similar products in the field. Section 3 focuses on the requirements gathering for this project, including a description of the various interviews we have conducted as well as the main requirements we identify. Section 4 then describes our chosen approach based on these requirements, including what programming language and tools we will use. Finally, Section 5 details what quality guarantees we will provide and how these are verified.